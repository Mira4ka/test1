\documentclass{article}
\usepackage[utf8]{inputenc}
\usepackage[english, russian]{babel}
\usepackage{amsmath}
\usepackage{indentfirst}
\usepackage{misccorr}

\begin{document}

\begin{center}
    {\it Гл. 1. Неопределенный интеграл} 
    \smallskip\hrule width 350 pt\smallskip
\end{center}

\par\smallskip 5) $ \displaystyle \int \dfrac{dx}{{\ch^3 x} + 3\hspace{2pt}{\ch x}} $; \hspace{10pt}
6) $ \displaystyle \int \dfrac{{\sh 2x} \hspace{2pt}dx}{({\sh x}+1)({\ch^2 x}-{\sh x})} $.

{\bf 15.} 1) $ \displaystyle \int \dfrac{{\tg x}\hspace{2pt}dx}{{\tg x}-3} $; \hspace{10pt}
2) $ \displaystyle \int \dfrac{{\cos x} - {\sin x}}{{\cos x}+{\sin x}} \hspace{2pt}dx $; \hspace{10pt}
3) $ \displaystyle \int \dfrac{{\sin \dfrac{x-a}{2}}}{{\sin \dfrac{x+a}{2}}} \hspace{2pt}dx $;

\par\smallskip 4) $ \displaystyle \int \dfrac{a_1\hspace{1pt}{\cos x}+b_1\hspace{1pt}{\sin x}}{a\hspace{1pt}{\cos x}+b\hspace{2pt}{\sin x}} \hspace{2pt}dx, \hspace{5pt} a^2+b^2 \ne 0 $;
\hspace{10pt} 5) $ \displaystyle \int \dfrac{4\hspace{2pt}{\ch x}-3\hspace{2pt}{\sh x}}{2\hspace{2pt}{\ch x}-{\sh x}} \hspace{2pt}dx $;

\par\smallskip 6) $ \displaystyle \int \dfrac{dx}{1-{\th x}} $; \hspace{10pt}
7) $ \displaystyle \int \dfrac{a_1\hspace{1pt}{\ch x}+b_1\hspace{1pt}{\sh x}}{a\hspace{1pt}{\ch x}+b\hspace{2pt}{\sh x}} \hspace{2pt}dx, \hspace{5pt} a^2+b^2 \ne 0 $.

\par\smallskip{\bf 16.} 1) $ \displaystyle \int \dfrac{dx}{2\hspace{2pt}{\cos^2x}+{\sin x}{\cos x}+{\sin^2x}} $; \hspace{10pt}
2) $ \displaystyle \int \dfrac{dx}{{\cos 2x}-{\sin 2x}} $;

\par\smallskip 3) $ \displaystyle \int \dfrac{dx}{4\hspace{2pt}{\cos^2x}-2\hspace{2pt}{\sin 2x}+{\sin^2x}} $; \hspace{10pt}
4) $ \displaystyle \int \dfrac{dx}{5+{\cos^2x}}, \hspace{5pt} |x| < \dfrac{\pi}{2} $;

\par\smallskip 5) $ \displaystyle \int \dfrac{dx}{2+3\hspace{2pt}{\sin 2x}-4\hspace{2pt}{\cos^2x}} $; \hspace{10pt}
6) $ \displaystyle \int \dfrac{dx}{a\hspace{1pt}{\cos^2x}+b\hspace{2pt}{\sin 2x}+c\hspace{2pt}{\sin^2x}}, \hspace{5pt}c>0 $;

\par\smallskip 7) $ \displaystyle \int \dfrac{dx}{3\hspace{2pt}{\sh^2x}-7\hspace{2pt}{\sh x}\hspace{1pt}{\ch x} +3\hspace{2pt}{\ch^2x}}$; \hspace{10pt}
8) $ \displaystyle \int \dfrac{dx}{10\hspace{2pt}{\ch^2x}-2\hspace{2pt}{\sh 2x}-1} $;

\par\smallskip 9) $ \displaystyle \int \dfrac{dx}{4+3\hspace{2pt}{\sh^2x}} $; \hspace{10pt}
10) $ \displaystyle \int \dfrac{dx}{1-6\hspace{2pt}{\sh 2x}-37\hspace{2pt}{\ch x}} $.

\par\smallskip{\bf 17.} 1) $ \displaystyle \int \dfrac{dx}{{\tg^2x}+4\hspace{2pt}{\tg x}} $; \hspace{10pt}
2) $ \displaystyle \int \dfrac{{\th x}\hspace{2pt}dx}{({\th x}+2)^2} $.

\par\smallskip{\bf 18.} 1) $ \displaystyle \int \dfrac{1+{\th x}}{{\sin 2x}}\hspace{2pt}dx $; \hspace{10pt}
2) $ \displaystyle \int \dfrac{{\cos x}\hspace{2pt}dx}{{\sin^3x}+{\cos^3x}} $; \hspace{10pt}
3) $ \displaystyle \int \dfrac{{\sin 2x}\hspace{2pt}dx}{{\sin^4x}+{\cos^4x}} $;

\par\smallskip 4) $ \displaystyle \int \dfrac{dx}{{\sin^4x}+{\cos^4x}} $; \hspace{10pt}
5) $ \displaystyle \int \dfrac{{\cos^2x}\hspace{2pt}dx}{(a^2\hspace{2pt}{\sin^2x}+b^2\hspace{2pt}{\cos^2x})^2},
\hspace{5pt} a^2+b^2 \ne 0 $;

\par\smallskip 6) $ \displaystyle \int \dfrac{dx}{{\sin^6x}+{\cos^6x}} $; \hspace{10pt}
7) $ \displaystyle \int \dfrac{{\sh^2x}\hspace{2pt}dx}{1-{\sh^2x}} $; \hspace{10pt}
8) $ \displaystyle \int \dfrac{{\ch 2x}\hspace{2pt}dx}{{\sh^4x}+{\ch^4x}} $;

\par\smallskip 9) $ \displaystyle \int \dfrac{{\sh 2x}\hspace{2pt}dx}{1+{\sh^4x}} $; \hspace{10pt}
10) $ \displaystyle \int \dfrac{dx}{({\ch 2x}+{\ch^2x})^2} $.

\par\smallskip{\bf 19.} 1) $ \displaystyle \int \dfrac{dx}{{\sin x}+{\cos x}} $; \hspace{10pt}
2) $ \displaystyle \int \dfrac{dx}{\sqrt{3}\hspace{2pt}{\cos x}+{\sin x}} $; \hspace{10pt}
3) $ \displaystyle \int \dfrac{dx}{{\sh x}+2\hspace{2pt}{\ch x}} $;

\par\smallskip 4) $ \displaystyle \int \dfrac{dx}{2\hspace{2pt}{\sh x}-{\ch x}} $; \hspace{10pt}
5) $ \displaystyle \int \dfrac{dx}{a\hspace{2pt}{\ch x}+b\hspace{2pt}{\sh x}}, \hspace{5pt} a>0 $; 

\par\smallskip 6) $ \displaystyle \int \dfrac{dx}{a\hspace{2pt}{\cos x}+b\hspace{2pt}{\sin x}}, \hspace{5pt} 
a^2+b^2 \ne 0 $.

\par\smallskip{\bf 20.} Для интеграла $ J_n = \displaystyle \int \dfrac{dx}{(a\hspace{2pt}{\sin x}+b\hspace{2pt}{\cos x})^n}, \hspace{5pt} a^2+b^2 \ne 0, \hspace{5pt} n \in {\bf N} $, \hspace{5pt} доказать рекуррентную формулу

\[
   J_n = \dfrac{1}{(n-1)(a^2+b^2)}\left(\dfrac{a\hspace{2pt}{\sin x}-b\hspace{2pt}{\cos x}}{(a\hspace{2pt}{\sin x}+
   b\hspace{2pt}{\cos x})^{n-1}}+(n-2)J_{n-2}\right), \hspace{5pt} n>1,
\]

и с ее помощью найти интеграл $ \displaystyle \int \dfrac{dx}{(2\hspace{2pt}{\cos x}+{\sin x})^3} $.

\par\smallskip{\bf 21.} Найти: 

\par\smallskip 1) $ \displaystyle \int \dfrac{dx}{1+4\hspace{2pt}{\cos x}} $; \hspace{10pt}
2) $ \displaystyle \int \dfrac{dx}{4+{\cos x}} $; \hspace{10pt}
3) $ \displaystyle \int \dfrac{dx}{4-{\sin x}} $;

\end{document}
